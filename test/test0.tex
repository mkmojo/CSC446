%%%%%%%%%%%%%%%%%%%%%%%%%%%%%%%%%%%%%%%%%
% Structured General Purpose Assignment
% LaTeX Template
%
% This template has been downloaded from:
% http://www.latextemplates.com
%
% Original author:
% Ted Pavlic (http://www.tedpavlic.com)
%
% Note:
% The \lipsum[#] commands throughout this template generate dummy text
% to fill the template out. These commands should all be removed when 
% writing assignment content.
%
%%%%%%%%%%%%%%%%%%%%%%%%%%%%%%%%%%%%%%%%%

%----------------------------------------------------------------------------------------
%	PACKAGES AND OTHER DOCUMENT CONFIGURATIONS
%----------------------------------------------------------------------------------------

\documentclass{article}

\usepackage{fancyhdr} % Required for custom headers
\usepackage{lastpage} % Required to determine the last page for the footer
\usepackage{extramarks} % Required for headers and footers
\usepackage{graphicx} % Required to insert images
\usepackage{lipsum} % Used for inserting dummy 'Lorem ipsum' text into the template
\usepackage{mathtools}
% Margins
\topmargin=-0.45in
\evensidemargin=0in
\oddsidemargin=0in
\textwidth=6.5in
\textheight=9.0in
\headsep=0.25in 

\linespread{1.1} % Line spacing

% Set up the header and footer
\pagestyle{fancy}
\lhead{\hmwkAuthorName} % Top left header
\chead{\hmwkClass\ \hmwkTitle} % Top center header
\rhead{\firstxmark} % Top right header
\lfoot{\lastxmark} % Bottom left footer
\cfoot{} % Bottom center footer
\rfoot{Page\ \thepage\ of\ \pageref{LastPage}} % Bottom right footer
\renewcommand\headrulewidth{0.4pt} % Size of the header rule
\renewcommand\footrulewidth{0.4pt} % Size of the footer rule

\setlength\parindent{0pt} % Removes all indentation from paragraphs

%----------------------------------------------------------------------------------------
%	DOCUMENT STRUCTURE COMMANDS
%	Skip this unless you know what you're doing
%----------------------------------------------------------------------------------------

% Header and footer for when a page split occurs within a problem environment
\newcommand{\enterProblemHeader}[1]{
\nobreak\extramarks{#1}{#1 continued on next page\ldots}\nobreak
\nobreak\extramarks{#1 (continued)}{#1 continued on next page\ldots}\nobreak
}

% Header and footer for when a page split occurs between problem environments
\newcommand{\exitProblemHeader}[1]{
\nobreak\extramarks{#1 (continued)}{#1 continued on next page\ldots}\nobreak
\nobreak\extramarks{#1}{}\nobreak
}

\setcounter{secnumdepth}{0} % Removes default section numbers
\newcounter{homeworkProblemCounter} % Creates a counter to keep track of the number of problems

\newcommand{\homeworkProblemName}{}
\newenvironment{homeworkProblem}[1][Problem \arabic{homeworkProblemCounter}]{ % Makes a new environment called homeworkProblem which takes 1 argument (custom name) but the default is "Problem #"
\stepcounter{homeworkProblemCounter} % Increase counter for number of problems
\renewcommand{\homeworkProblemName}{#1} % Assign \homeworkProblemName the name of the problem
\section{\homeworkProblemName} % Make a section in the document with the custom problem count
\enterProblemHeader{\homeworkProblemName} % Header and footer within the environment
}{
\exitProblemHeader{\homeworkProblemName} % Header and footer after the environment
}

\newcommand{\problemAnswer}[1]{ % Defines the problem answer command with the content as the only argument
\noindent\framebox[\columnwidth][c]{\begin{minipage}{0.98\columnwidth}#1\end{minipage}} % Makes the box around the problem answer and puts the content inside
}

\newcommand{\homeworkSectionName}{}
\newenvironment{homeworkSection}[1]{ % New environment for sections within homework problems, takes 1 argument - the name of the section
\renewcommand{\homeworkSectionName}{#1} % Assign \homeworkSectionName to the name of the section from the environment argument
\subsection{\homeworkSectionName} % Make a subsection with the custom name of the subsection
\enterProblemHeader{\homeworkProblemName\ [\homeworkSectionName]} % Header and footer within the environment
}{
\enterProblemHeader{\homeworkProblemName} % Header and footer after the environment
}
   
%----------------------------------------------------------------------------------------
%	NAME AND CLASS SECTION
%----------------------------------------------------------------------------------------

\newcommand{\hmwkTitle}{Assignment\ \#2} % Assignment title
\newcommand{\hmwkDueDate}{Wed,\ Jan.\ 29,\ 2014} % Due date
\newcommand{\hmwkClass}{CSC\ 446} % Course/class
\newcommand{\hmwkClassTime}{} % Class/lecture time
\newcommand{\hmwkClassInstructor}{} % Teacher/lecturer
\newcommand{\hmwkAuthorName}{Qiyuan Qiu} % Your name

%----------------------------------------------------------------------------------------
%	TITLE PAGE
%----------------------------------------------------------------------------------------

\title{
\vspace{2in}
\textmd{\textbf{\hmwkClass:\ \hmwkTitle}}\\
\normalsize\vspace{0.1in}\small{Due\ on\ \hmwkDueDate}\\
\vspace{0.1in}\large{\textit{\hmwkClassInstructor\ \hmwkClassTime}}
\vspace{3in}
}

\author{\textbf{\hmwkAuthorName}}
\date{} % Insert date here if you want it to appear below your name

%----------------------------------------------------------------------------------------

\begin{document}

\maketitle

%----------------------------------------------------------------------------------------
%	TABLE OF CONTENTS
%----------------------------------------------------------------------------------------

%\setcounter{tocdepth}{1} % Uncomment this line if you don't want subsections listed in the ToC

%\newpage
%\tableofcontents

\newpage

%----------------------------------------------------------------------------------------
%	PROBLEM 1
%----------------------------------------------------------------------------------------

% To have just one problem per page, simply put a \clearpage after each problem

\begin{homeworkProblem}
    %\textbf{x}
    x
    \textbf{X}
    %---------------------------------
    \begin{homeworkSection}{(a)}
    \problemAnswer{ % Answer
    \begin{equation*}
        \begin{split}
        H(Y|X)  &= -\sum\limits_{x,y}p(x,y)logP(y|x) \\
                &= -\sum\limits_{x,y}p(x,y)log\frac{P(x,y)}{P(x)} \\
                &= -\sum\limits_{x,y}p(x,y)(logP(x,y) - logP(x)) \\
                &= -\sum\limits_{x,y}p(x,y)logP(x,y) - -\sum\limits_{x,y}p(x,y)logP(x) \\
                &= -\sum\limits_{x,y}p(x,y)logP(x,y) - -\sum\limits_{x}p(x)logP(x) \\
                &= H(X,Y) - H(X)
        \end{split}
    \end{equation*}
    Therefore:
    $$H(X,Y) = H(X) + H(Y|X)$$
    }
    \end{homeworkSection}
    %---------------------------------
    \begin{homeworkSection}{(b)}
    \problemAnswer{ % Answer
    \begin{equation*}
        \begin{split}
            H(X) - H(X|Y)   &= -\sum\limits_{x}p(x)logP(x) + 
                            \sum\limits_{x,y}p(x,y)log\frac{P(x,y)}{P(y)} \\
                            &= \sum\limits_{x}p(x)log\frac{1}{P(x)} +
                            \sum\limits_{x,y}p(x,y)log\frac{P(x,y)}{P(y)} \\
                            &= \sum\limits_{x,y}p(x,y)log\frac{1}{P(x)} +
                            \sum\limits_{x,y}p(x,y)log\frac{P(x,y)}{P(y)} \\
                            &= \sum\limits_{x,y}p(x,y)log\frac{P(x,y)}{P(x)P(y)} \\ 
                            &= I(X;Y)
        \end{split}
    \end{equation*}
    Therefore:
    $$I(X; Y) = H(X) - H(X|Y)$$
    }
    \end{homeworkSection}
    %---------------------------------
    \begin{homeworkSection}{(c)}
    \problemAnswer{ % Answer
        From (a) we know that 
        \begin{equation}
            \label{eq:1}
        H(X,Y) = H(X) + H(Y|X)
        \end{equation}
        \begin{equation}
            \label{eq:2}
        H(Y,X) = H(Y) + H(X|Y)
        \end{equation}
        ~\eqref{eq:1} - ~\eqref{eq:2}, we have
        $$0 = H(X) - H(Y) + H(Y|Z) - H(X|Y)$$
        with a a little bit of rearrangement:
        $$H(X) - H(X|Y) = H(Y) - H(Y|X)$$
    }
    \end{homeworkSection}
    %---------------------------------
    \begin{homeworkSection}{(d)}
    \problemAnswer{ % Answer
        From (a) we know
        $$H(X,Y) = H(X) + H(Y|X)$$
        \begin{equation*}
        \begin{split}
            H(X) + H(Y) &\ge H(X) + H(Y|X) \\
            H(Y) &\ge H(Y|X) \\
            I(Y;X) &\ge 0 \text{  which is always true} 
        \end{split}
        \end{equation*}
        Therefore we have 
        $$H(X) + H(Y) \ge H(X,Y)$$
    }
    \end{homeworkSection}
\end{homeworkProblem}

%----------------------------------------------------------------------------------------
%	PROBLEM 2
%----------------------------------------------------------------------------------------

\begin{homeworkProblem}

    %---------------------------------
    \begin{homeworkSection}{(a)}
    \problemAnswer{ % Answer
    $$L(\vec{x}, \lambda) =  \| \vec{x} \|^2  + \lambda(\sum\limits_i \vec{x} - 1)$$
    which is equivalent to
        \begin{equation}
            \begin{cases}
                &\frac{\partial L}{\partial \vec{x}} = 0 \\
                &\sum \limits_i x_i - 1 = 0
            \end{cases}
        \end{equation}
    which is equivalent to 
        \begin{equation}
            \begin{cases}
                &\frac{\partial L}{\partial x_i} = 0 \\
                &\sum \limits_i x_i - 1 = 0
            \end{cases}
        \end{equation}
    which is equivalent to 
        \begin{equation}
            \begin{cases}
                & x_i = \frac{- \lambda}{2} \\
                &\sum \limits_i x_i - 1 = 0
            \end{cases}
        \end{equation}
    Therefore 
    \begin{equation}
    \begin{cases}
        & \lambda = \frac{-2}{n} \\
        & x_i = \frac{1}{n}
    \end{cases}
    \end{equation}
    }
    \end{homeworkSection}

%---------------------------------
    \begin{homeworkSection}{(b)}
    \problemAnswer{ % Answer
        $$L(\vec{x}, \lambda) = \sum\limits_i x_i + 
                \lambda\left(\sum\limits_i x_i^2 - 1\right)$$
        which is equivalent to
            \begin{equation}
                \begin{cases}
                    &\frac{\partial L}{\partial \vec{x}} = 0 \\
                    &\sum \limits_i x_i^2 - 1 = 0
                \end{cases}
            \end{equation}
        which is equivalent to
            \begin{equation}
                \label{eq:p2b}
                \begin{cases}
                    &1 + 2\lambda x_i = 0 \\
                    &\sum \limits_i x_i^2 - 1 = 0
                \end{cases}
            \end{equation}
        solve ~\eqref{eq:p2b}, we get:
        \begin{equation}
            \begin{cases}
                &\lambda = \frac{\sqrt{n}}{2} \\
                &x_i = \frac{-1}{\sqrt{n}}
            \end{cases}
        \end{equation}
    }
    \end{homeworkSection}
%---------------------------------
    \begin{homeworkSection}{(c)}
    \problemAnswer{ % Answer
        $$L(Q, \lambda) = \sum\limits_i \ln \left( \frac{P(i)}{Q(i)}\right) P(i)- 
        \lambda\left(\sum\limits_i Q(i) - 1\right)$$
        which is equivalent to
            \begin{equation}
                \begin{cases}
                    &\frac{\partial L}{\partial Q(i)} = 0 \\
                    &\sum\limits_i Q(i) - 1 = 0
                \end{cases}
            \end{equation}
        which is equivalent to
            \begin{equation}
                \label{eq:p2c}
                \begin{cases}
                    &Q(i) = \frac{-P(i)}{\lambda} \\
                    &\sum \limits_i Q(i) - 1 = 0
                \end{cases}
            \end{equation}
        solve ~\eqref{eq:p2c}, we get:
        \begin{equation}
            \begin{cases}
                &\lambda = -1 \\
                &Q(i) = P(i)
            \end{cases}
        \end{equation}
    }
    \end{homeworkSection}
%---------------------------------
    \begin{homeworkSection}{(d)}
    \problemAnswer{ % Answer
        $$L(Q, \lambda) = -\sum\limits_k Q(k)\ln{Q(k)} + 
        \lambda\left(\sum\limits_k Q(k) - 1\right)$$
        which is equivalent to
            \begin{equation}
                \begin{cases}
                    &\frac{\partial L}{\partial Q(k)} = 0 \\
                    &\sum\limits_i Q(k) - 1 = 0
                \end{cases}
            \end{equation}
        which is equivalent to
            \begin{equation}
                \label{eq:p2d}
                \begin{cases}
                    &Q(k) = e^{\lambda - 1} \\
                    &\sum\limits_k Q(k) - 1 = 0
                \end{cases}
            \end{equation}

        solve ~\eqref{eq:p2d}, we get:
        \begin{equation}
            \begin{cases}
                &\lambda = 1 - \ln K \\
                &Q(k) = \frac{1}{K} 
            \end{cases}
        \end{equation}
    }
    \end{homeworkSection}
%%---------------------------------
\end{homeworkProblem}

%----------------------------------------------------------------------------------------

\end{document}
