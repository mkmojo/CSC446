%%%%%%%%%%%%%%%%%%%%%%%%%%%%%%%%%%%%%%%%%
% Structured General Purpose Assignment
% LaTeX Template
%
% This template has been downloaded from:
% http://www.latextemplates.com
%
% Original author:
% Ted Pavlic (http://www.tedpavlic.com)
%
% Note:
% The \lipsum[#] commands throughout this template generate dummy text
% to fill the template out. These commands should all be removed when 
% writing assignment content.
%
%%%%%%%%%%%%%%%%%%%%%%%%%%%%%%%%%%%%%%%%%

%----------------------------------------------------------------------------------------
%	PACKAGES AND OTHER DOCUMENT CONFIGURATIONS
%----------------------------------------------------------------------------------------

\documentclass{article}

\usepackage{fancyhdr} % Required for custom headers
\usepackage{lastpage} % Required to determine the last page for the footer
\usepackage{extramarks} % Required for headers and footers
\usepackage{graphicx} % Required to insert images
\usepackage{lipsum} % Used for inserting dummy 'Lorem ipsum' text into the template
\usepackage{mathtools}
% Margins
\topmargin=-0.45in
\evensidemargin=0in
\oddsidemargin=0in
\textwidth=6.5in
\textheight=9.0in
\headsep=0.25in 

\linespread{1.1} % Line spacing

% Set up the header and footer
\pagestyle{fancy}
\lhead{\hmwkAuthorName} % Top left header
\chead{\hmwkClass\ \hmwkTitle} % Top center header
\rhead{\firstxmark} % Top right header
\lfoot{\lastxmark} % Bottom left footer
\cfoot{} % Bottom center footer
\rfoot{Page\ \thepage\ of\ \pageref{LastPage}} % Bottom right footer
\renewcommand\headrulewidth{0.4pt} % Size of the header rule
\renewcommand\footrulewidth{0.4pt} % Size of the footer rule

\setlength\parindent{0pt} % Removes all indentation from paragraphs

%----------------------------------------------------------------------------------------
%	DOCUMENT STRUCTURE COMMANDS
%	Skip this unless you know what you're doing
%----------------------------------------------------------------------------------------

% Header and footer for when a page split occurs within a problem environment
\newcommand{\enterProblemHeader}[1]{
\nobreak\extramarks{#1}{#1 continued on next page\ldots}\nobreak
\nobreak\extramarks{#1 (continued)}{#1 continued on next page\ldots}\nobreak
}

% Header and footer for when a page split occurs between problem environments
\newcommand{\exitProblemHeader}[1]{
\nobreak\extramarks{#1 (continued)}{#1 continued on next page\ldots}\nobreak
\nobreak\extramarks{#1}{}\nobreak
}

\setcounter{secnumdepth}{0} % Removes default section numbers
\newcounter{homeworkProblemCounter} % Creates a counter to keep track of the number of problems

\newcommand{\homeworkProblemName}{}
\newenvironment{homeworkProblem}[1][Problem \arabic{homeworkProblemCounter}]{ % Makes a new environment called homeworkProblem which takes 1 argument (custom name) but the default is "Problem #"
\stepcounter{homeworkProblemCounter} % Increase counter for number of problems
\renewcommand{\homeworkProblemName}{#1} % Assign \homeworkProblemName the name of the problem
\section{\homeworkProblemName} % Make a section in the document with the custom problem count
\enterProblemHeader{\homeworkProblemName} % Header and footer within the environment
}{
\exitProblemHeader{\homeworkProblemName} % Header and footer after the environment
}

\newcommand{\problemAnswer}[1]{ % Defines the problem answer command with the content as the only argument
\noindent\framebox[\columnwidth][c]{\begin{minipage}{0.98\columnwidth}#1\end{minipage}} % Makes the box around the problem answer and puts the content inside
}

\newcommand{\homeworkSectionName}{}
\newenvironment{homeworkSection}[1]{ % New environment for sections within homework problems, takes 1 argument - the name of the section
\renewcommand{\homeworkSectionName}{#1} % Assign \homeworkSectionName to the name of the section from the environment argument
\subsection{\homeworkSectionName} % Make a subsection with the custom name of the subsection
\enterProblemHeader{\homeworkProblemName\ [\homeworkSectionName]} % Header and footer within the environment
}{
\enterProblemHeader{\homeworkProblemName} % Header and footer after the environment
}
   
%----------------------------------------------------------------------------------------
%	NAME AND CLASS SECTION
%----------------------------------------------------------------------------------------

\newcommand{\hmwkTitle}{Assignment\ \#1} % Assignment title
\newcommand{\hmwkDueDate}{Wed,\ Jan.\ 21,\ 2014} % Due date
\newcommand{\hmwkClass}{CSC\ 446} % Course/class
\newcommand{\hmwkClassTime}{} % Class/lecture time
\newcommand{\hmwkClassInstructor}{} % Teacher/lecturer
\newcommand{\hmwkAuthorName}{Qiyuan Qiu} % Your name

%----------------------------------------------------------------------------------------
%	TITLE PAGE
%----------------------------------------------------------------------------------------

\title{
\vspace{2in}
\textmd{\textbf{\hmwkClass:\ \hmwkTitle}}\\
\normalsize\vspace{0.1in}\small{Due\ on\ \hmwkDueDate}\\
\vspace{0.1in}\large{\textit{\hmwkClassInstructor\ \hmwkClassTime}}
\vspace{3in}
}

\author{\textbf{\hmwkAuthorName}}
\date{} % Insert date here if you want it to appear below your name

%----------------------------------------------------------------------------------------

\begin{document}

\maketitle

%----------------------------------------------------------------------------------------
%	TABLE OF CONTENTS
%----------------------------------------------------------------------------------------

%\setcounter{tocdepth}{1} % Uncomment this line if you don't want subsections listed in the ToC

%\newpage
%\tableofcontents

\newpage

%----------------------------------------------------------------------------------------
%	PROBLEM 1
%----------------------------------------------------------------------------------------

% To have just one problem per page, simply put a \clearpage after each problem

\begin{homeworkProblem}
\vspace{10pt} % Question

\problemAnswer{ % Answer

$$E[XY] = \sum\limits_{x=-\infty}^{+\infty}\sum\limits_{y=-\infty}^{+\infty} xyp_{X,Y}(x, y)$$
bacause $X$ and $Y$ are independent, 
$$p_{X,Y}(x, y) = p_{X}(x) p_{Y}(y)$$
therefore, 
\begin{equation*}
\begin{split}
\sum\limits_{x=-\infty}^{+\infty}\sum\limits_{y=-\infty}^{+\infty} xyp_{X,Y}(x, y) &= \sum\limits_{x=-\infty}^{+\infty}\sum\limits_{y=-\infty}^{+\infty} xyp_{X}(x) p_{Y}(y) \\ 
&= \sum\limits_{x=-\infty}^{+\infty}xp_{X}(x)\sum\limits_{y=-\infty}^{+\infty}yp_{Y}(y) \\
& = E[X]E[Y]
\end{split}
\end{equation*}
}
\end{homeworkProblem}

%----------------------------------------------------------------------------------------
%	PROBLEM 2
%----------------------------------------------------------------------------------------


\begin{homeworkProblem}
\vspace{10pt} % Question

\problemAnswer{ % Answer
\begin{equation*}
\begin{split}
P(|X-\mu|^2 \ge k^2\sigma^2)
 & = P((X-\mu)^2 \ge k^2\sigma^2) \\
 & \le \frac{E[(X-\mu)^2]}{k^2\sigma^2} \\
 & = \frac{\sigma^2}{k^2\sigma^2} \\
 & = \frac{1}{k^2}
\end{split}
\end{equation*}
}
\end{homeworkProblem}

%----------------------------------------------------------------------------------------
%	PROBLEM 3
%----------------------------------------------------------------------------------------

\begin{homeworkProblem}
%--------------------------------------------
\begin{homeworkSection}{(a)} % Section within problem
\vspace{10pt} % Question

\problemAnswer{ % Answer
$$E[R] = E[r_1\cdot r_2 \cdot r_3 \cdot \ldots \cdot r_N]$$ 
because $r_i$ is i.i.d, 
\begin{equation}
\label{eq:E_R}
E[R] = (E[r_1])^N 
\end{equation}
\begin{equation*}
\begin{split}
E[r_1] &= \sum\limits_{x=-\infty}^{+\infty}xp_{X}(x) \\
& = \frac{1}{2}\times \frac{1}{2} + 2 \times \frac{1}{2} \\
& = \frac{5}{4}
\end{split}
\end{equation*}
Therefore because of ~\eqref{eq:E_R}
$$E[R] = \left(\frac{5}{4}\right)^N$$
}
\end{homeworkSection}

%--------------------------------------------

\begin{homeworkSection}{(b)} % Section within problem
\problemAnswer{ % Answer
$X$ could be seen as doing the same experiment for $N$ times. 
Therefor the relationship between $X$ and $x_i$ should be similar to that between Binomial and Bernoulli. 
\begin{equation*}
\begin{split}
E[x_1] &= \sum\limits_{x=-\infty}^{+\infty}xp_{X_{1}}(x) \\ 
&= log\frac{1}{2} \times \frac{1}{2} + log2 \times \frac{1}{2} \\
&=0
\end{split}
\end{equation*}
Hence, 
\begin{equation}
\label{eq:E_X}
E[X] = N \times E[x_1] = 0
\end{equation}
}
\end{homeworkSection}

%--------------------------------------------

\begin{homeworkSection}{(c)} % Section within problem
\problemAnswer{ % Answer
$$Var[X_i] = Var[X_1] = E[X_1^2] - \left( E[X_1]\right)^2$$
\begin{equation*}
\begin{split}
E[X_{1}^2]  &= \sum\limits_{x=-\infty}^{+\infty}x_1^2p_{X_{1}}(x) - \left( E[X_1]\right)^2 \\
& = \left(log\frac{1}{2}\right)^2\times\frac{1}{2} + \left(log2\right)^2\times \frac{1}{2} \\
& = log^2 2
\end{split}
\end{equation*}

Because $E[X_1]$ = 0, so  
$$Var[X_i] = E[X_{1}^2] =  log^2 2$$

}
\end{homeworkSection}

%--------------------------------------------

\begin{homeworkSection}{(d)} % Section within problem
\problemAnswer{ % Answer
Because $x_i$ are i.i.d. \\
So, 
$$Var[X] = NVar[X_1] = Nlog^2 2$$
}
\end{homeworkSection}

%--------------------------------------------

\begin{homeworkSection}{(e)} % Section within problem
\problemAnswer{ % Answer
We want to know
$$P\left(X > \left(\frac{9}{8}\right)^N\right)$$
We already know from ~\eqref{eq:E_X} that
$$\mu = E[X] = 0$$
$$P\left(X > \left(\frac{9}{8}\right)^N\right) = P\left(X-\mu > \left(\frac{9}{8}\right)^N\right)$$
Because $X > 0$, $\mu = 0$ so $X-\mu$ = $|X-\mu|$
So
\begin{equation*}
\begin{split}
P\left(X > \left(\frac{9}{8}\right)^N\right)& = P\left(|X-\mu| > \left(\frac{9}{8}\right)^N\right) \\
& \le \frac{E((X-\mu)^2)}{\left(\frac{9}{8}\right)^{2N}} \\
& = \frac{Nlog^2 2}{\left(\frac{9}{8}\right)^{2N}} \\
& = \lim_{N \to +\infty} \frac{Nlog^2 2}{\left(\frac{9}{8}\right)^{2N}}\\
& = 0
\end{split}
\end{equation*}
}
\end{homeworkSection}

%--------------------------------------------
\end{homeworkProblem}


%----------------------------------------------------------------------------------------
%	PROBLEM 4
%----------------------------------------------------------------------------------------
\begin{homeworkProblem}
%--------------------------------------------
\begin{homeworkSection}{(a)} % Section within problem
\problemAnswer{ % Answer
$$ r_i 
= \left\{
\begin{array}{l l}
    \frac{1}{4}& \quad \text{with prob $\frac{1}{4}$} \\
    \frac{5}{4}& \quad \text{with prob $\frac{1}{2}$} \\
    2 & \quad \text{with prob $\frac{1}{4}$} 
\end{array} 
\right.
$$
}
\end{homeworkSection}

%--------------------------------------------
\begin{homeworkSection}{(b)} % Section within problem
\problemAnswer{ % Answer
\begin{equation*}
\begin{split}
E[r_1] &= \sum\limits_{x = -\infty}^{+\infty}xp_X(r) \\
& = \frac{1}{2} \times \frac{1}{4} + \frac{5}{4} \times \frac{1}{2} + 2 \times \frac{1}{4} \\
& = \frac{5}{4}
\end{split}
\end{equation*}

For similar reason as in problem \#3, 
$$E[R] = \left(\frac{5}{4}\right)^N$$
}
\end{homeworkSection}
%--------------------------------------------
\begin{homeworkSection}{(c)} % Section within problem
\problemAnswer{ % Answer
\begin{equation*}
\begin{split}
E(x_1) &= log\left(\frac{1}{2}\right)\times\frac{1}{4} +  log\left(\frac{5}{4}\right)\times\frac{1}{2} +  log\left(2\right)\times\frac{1}{4} \\
& = \frac{1}{2}log\left(\frac{5}{4}\right) 
\end{split}
\end{equation*}

\begin{equation*}
\begin{split}
E(X) &= N \times \frac{1}{2}log\left(\frac{5}{4}\right)\\
& =  \frac{N}{2}log\left(\frac{5}{4}\right)
\end{split}
\end{equation*}
}
\end{homeworkSection}
%--------------------------------------------
\begin{homeworkSection}{(d)} % Section within problem
\problemAnswer{ % Answer
$$Var[X_i] = Var[X_1] = E[X_1^2] - \left(E[X_1]\right)^2$$

\begin{equation*}
\begin{split}
E[X_1^2] &= \sum\limits_{x_1 = -\infty}^{+\infty}x_1^2p_{X_1}(x_1) \\
& = log^2\left(\frac{1}{2}\right) \times \frac{1}{4} + log^2\left(\frac{5}{4}\right) \times \frac{1}{2} + log^2\left(2\right) \times \frac{1}{4}  \\
& = \frac{1}{2}log^2\left(2\right) + \frac{1}{2}log^2\left(\frac{5}{4}\right) 
\end{split}
\end{equation*}

\begin{equation*}
\begin{split}
Var[X_i] &= \frac{1}{2}log^2\left(2\right) + \frac{1}{2}log^2\left(\frac{5}{4}\right) - 
\left(\frac{1}{2}log\left(\frac{5}{4}\right)\right)^2 \\
& = \frac{1}{2}log^2\left(2\right) + \frac{1}{4}log^2\left(\frac{5}{4}\right)
\end{split}
\end{equation*}
}
\end{homeworkSection}
%--------------------------------------------
\begin{homeworkSection}{(e)} % Section within problem
\problemAnswer{ % Answer
$$Var[X] = N\left( \frac{1}{2}log^2\left(2\right) + \frac{1}{4}log^2\left(\frac{5}{4}\right) \right) $$

}
\end{homeworkSection}
%--------------------------------------------
\begin{homeworkSection}{(f)} % Section within problem
\problemAnswer{ % Answer
$$ P\left(X < \left( \frac{9}{8} \right)^N\right)  = 1 - P\left(X \ge \left( \frac{9}{8} \right)^N\right) $$
From the right hand side part, 
\begin{equation*}
\begin{split}
P\left(X \ge \left( \frac{9}{8} \right)^N\right) &= P\left(|X - \mu|^2 \ge \left(\left(\frac{9}{8}\right)^N - \mu\right)^2\right) \\
& \le \frac{E\left(|X - \mu|^2\right) = Var\left(X\right)}{\left( \left( \frac{9}{8} \right) - \mu \right)^2} \\
& = \frac{N\left(\frac{1}{2}log^2\left(2\right) + \frac{1}{4}log^2\left(\frac{5}{4}\right)\right)}{\left(\left(\frac{9}{8}\right)^N - \frac{N}{2}log\left(\frac{5}{4}\right)\right)^2} \\
& = \frac{\frac{1}{2}log^2\left(2\right) +\frac{1}{4}log^2\left(\frac{5}{4}\right)}{\frac{\left(\frac{9}{8}\right)^N}{N} + \frac{\frac{1}{4}log^2\left(\frac{5}{4}\right)}{\left(\frac{9}{8}\right)^N} - log^2\left(\frac{5}{4}\right)} 
\end{split}
\end{equation*}
$$\lim_{N \to +\infty}\frac{\frac{1}{2}log^2\left(2\right) +\frac{1}{4}log^2\left(\frac{5}{4}\right)}{\frac{\left(\frac{9}{8}\right)^N}{N} + \frac{\frac{1}{4}log^2\left(\frac{5}{4}\right)}{\left(\frac{9}{8}\right)^N} - log^2\left(\frac{5}{4}\right)} = 0$$
So when $N \rightarrow +\infty$ 
$$P\left(X \ge \left( \frac{9}{8} \right)^N\right) = P\left(|X - \mu|^2 \ge \left(\left(\frac{9}{8}\right)^N - \mu\right)^2\right) = 0 $$
So 
$$ P\left(X < \left( \frac{9}{8} \right)^N\right) = 1$$
}
\end{homeworkSection}
%--------------------------------------------
\begin{homeworkSection}{(g)} % Section within problem
\problemAnswer{ % Answer
For these two methods, their $E[R]$ are the same.
Bue the variance of problem \#4 is less than that of problem \#3. 
\begin{equation*}
\begin{split}
 Nlog^22 - N\left(\frac{1}{2}log^22 + \frac{1}{4}log^2(\frac{5}{4})\right)
& = \frac{N}{2}log^22 - \frac{N}{4}log^2(\frac{5}{4})\\
& \ge \frac{N}{4}log^22 - \frac{N}{4}log^2(\frac{5}{4}) \\
& > 0
\end{split}
\end{equation*}
This means if we choose the second method, we are more likely to be close to expectation. 
Therefore I will choose the latter one. 

}
\end{homeworkSection}
%--------------------------------------------
\end{homeworkProblem}

%----------------------------------------------------------------------------------------

\end{document}
